\documentclass[11pt,a4paper,english]{article}
\usepackage[utf8]{inputenc}
\usepackage{babel}
\usepackage[T1]{fontenc}
\usepackage{lmodern}
\usepackage[round]{natbib}
\usepackage[margin=1in]{geometry}  % set the margins to 1in on all sides
\usepackage{hyperref}
\usepackage{amsmath}
\usepackage{amssymb}
\usepackage{amsthm}
\usepackage{bm}

\newcommand{\PP}{\mathbb{P}}      % for Prob
\newcommand{\EE}{\mathbb{E}}      % for Expectation

\begin{document}
\title{Discrete Optimization Assignment 2}
\author{Quintin Hill, Chi Pham, William Sprent}
\maketitle
\tableofcontents
\clearpage

\section{Theoretical part}
The Randomized Rounding algorithm picks a collection $\mathcal{C}$ of sets to include in
the final solution. We wish to find the probability that at least $\frac{n}{2}$ sets and
the total cost of the sets $c(\mathcal{C})$ is less than $O(OPT)$.

We will equivalently bound the probability that the number of uncovered elements, $|NC|$,
exceeds $\frac{n}{2}$, and that $c(\mathcal{C}) \geq O(OPT_f)$. We will bound these seperately
and combine them by union bound.

\paragraph{Case 1 ( $c(\mathcal{C}) \geq O(OPT_f)$):}
We note that we have,
$$\EE[\mathrm{cost}(\mathcal{C})] = \mathrm{OPT}_f\mathnormal{.}$$
By applying Markov's inequality, we get
\begin{align*}
\PP[\mathrm{cost}(\mathcal{C}) \geq O(\mathrm{OPT}_f)]
  \leq \frac{\mathrm{OPT}_f}{O(\mathrm{OPT}_f)}
  = \frac{1}{c}
\end{align*}
for some constant, $c$.

\paragraph{Case 2 ($|NC| \geq \frac{n}{2}$):}
We have that the probability that an element, $a$, is uncovered, $a \in NC$, is the following
$$\PP[a \in NC] \leq \frac{1}{e}$$
\citep[p. 121]{Vaz}.
Then we have that the expected number of uncovered elements is the following,
$$\EE[|NC|] = \sum_a \PP[a \in NC] \leq \frac{n}{e}\mathnormal{.}$$
Then we can derive a bound, by applying Markov's inequality again,
\begin{align*}
  \PP[|NC| \geq \frac{n}{2}] \leq \frac{\EE[|NC|]}{\frac{n}{2}} 
                             \leq \frac{\frac{n}{e}}{\frac{n}{2}} 
                             = \frac{2}{e}\mathnormal{.}
\end{align*}

\paragraph{Combined Bound}
We then combine the bounds by union bound which gives us
$$\PP\left[\mathrm{cost}(\mathcal{C}) \geq O(\mathrm{OPT}_f) \vee  |NC| \geq \frac{n}{2} \right]
\leq \frac{1}{c} + \frac{2}{e}$$
for some constant $c$.

\clearpage


\section{Implementation part}

We report the best Solutions in Table 1 and Computational Time in Table 2 against each of the instances for each of the methods. For the two tables, n represents number of times run. As well Simplified Rounding is represented by (SR), Randomized Rounding by (RR), Primal-Dual Method(PDM), and Integer Linear Problem(ILP)

\paragraph{Solutions}\mbox{}\\


\begin{table}[h!]
  \centering
  \begin{tabular}{|c|c|c|c|c|c|c|}\hline
    Instance& $n$& CPLEX&SR&RR&PDM&ILP  \\\hline
    scpa3.dat &$0$&$0$ & $0$ & $0$& $0$ & $0$  \\
    scpc3.dat &$0$&$0$ & $0$ & $0$& $0$ & $0$ \\
    scpnrf1.dat &$0$&$0$ & $0$ & $0$& $0$ & $0$  \\
    scpnrg5.dat &$0$&$0$ & $0$ & $0$& $0$ & $0$  \\\hline
  \end{tabular}
  \caption{Results of methods on their Solution Value}
  \label{tab:res}
\end{table}

\paragraph{Computational Time}\mbox{}\\

\begin{table}[h!]
  \centering
  \begin{tabular}{|c|c|c|c|c|c|c|}\hline
    Instance& $n$& CPLEX&SR&RR&PDM&ILP \\\hline
    scpa3.dat &$0$&$0$ & $0$ & $0$& $0$ & $0$  \\
    scpc3.dat &$0$&$0$ & $0$ & $0$& $0$ & $0$ \\
    scpnrf1.dat	 &$0$&$0$ & $0$ & $0$& $0$ & $0$  \\
    scpnrg5.dat &$0$&$0$ & $0$ & $0$& $0$ & $0$  \\\hline
  \end{tabular}
  \caption{Results for Computational Time in ms}
  \label{tab:res}
\end{table}

\paragraph{How each Method relates to the others}\mbox{}\\
We note that with CPLEX ...



\subsection{CPLEX specific Analysis}
if necessary
\subsection{Rounding}
bla bla and
\paragraph{Relate the Randomized Rounding to the theoretical question addressed in 1.1.}

\subsection{Primal-Dual Method specific Analysis}
if necessary



cite wolsey here


\clearpage
\bibliographystyle{abbrvnat}
\bibliography{./lit.bib}
% bibtex do-2
\end{document}